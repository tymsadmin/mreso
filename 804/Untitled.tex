\documentclass[a4paper,11pt]{article}
\usepackage[utf8]{inputenc}
\usepackage[T1]{fontenc}
\usepackage[french]{babel}
\usepackage{lmodern}
\usepackage{amsmath,amssymb}
\usepackage{geometry}
\setlength{\parindent}{0pt}
\setlength{\footskip}{1.5cm}
\usepackage{XCharter}

\geometry{
  top=2cm,
  bottom=1.5cm,
  left=4cm,
  right=4cm,
  includefoot
}

\begin{document}

\section*{Questions du premier TP}

\subsection*{Q1}
\textbf{Question :} Afficher le pseudo des utilisateurs et l’année de création de leur compte Twitter.\\
\textbf{Nombre de lignes attendues :} 49

\subsection*{Q2}
\textbf{Question :} Lister les années de création (ANNEECR) des comptes utilisateurs (en éliminant les doublons).\\
\textbf{Nombre de lignes attendues :} 7

\subsection*{Q3}
\textbf{Question :} Afficher le pseudo et la biographie des utilisatrices. Renommer les en-têtes de colonnes pour plus de lisibilité.\\
\textbf{Nombre de lignes attendues :} 21

\subsection*{Q4}
\textbf{Question :} Modifier la requête précédente pour concaténer le résultat en une seule colonne.\\
\textbf{Nombre de lignes attendues :} 24

\subsection*{Q5}
\textbf{Question :} Lister les communes recensées dans la base, triées par code postal croissant puis par nom de commune (ordre alphabétique inverse en cas de codes postaux ex æquo).\\
\textbf{Nombre de lignes attendues :} 28

\subsection*{Q6}
\textbf{Question :} Afficher les pseudos des 19 utilisateurs situés entre \textit{BaeCutie} et \textit{DrasSnow} dans l’ordre alphabétique.\\
\textbf{Nombre de lignes attendues :} 19

\subsection*{Q7}
\textbf{Question :} Afficher les communes (dans l’ordre alphabétique) dont le code postal est situé entre 2 valeurs demandées à l’exécution de la requête (variables de substitution).\\
\textbf{Nombre de lignes attendues :} 21

\subsection*{Q8}
\textbf{Question :} Afficher les abonnements (\textit{followed}, \textit{follower}) qui ont débuté en août 2021.\\
\textbf{Nombre de lignes attendues :} 11

\subsection*{Q9}
\textbf{Question :} Lister les comptes utilisateurs (IDU) créés en 2015, 2017 et 2019.\\
\textbf{Nombre de lignes attendues :} 23

\subsection*{Q10}
\textbf{Question :} Afficher les tweets dont le texte contient le terme « désolé ».\\
\textbf{Nombre de lignes attendues :} 4

\subsection*{Q11}
\textbf{Question :} Afficher les utilisateurs dont la biographie n’est pas fournie.\\
\textbf{Nombre de lignes attendues :} 8

\subsection*{Q12}
\textbf{Question :} Donner le nombre de tweets enregistrés dans la base.\\
\textbf{Nombre de lignes attendues :} 1

\subsection*{Q13}
\textbf{Question :} Donner le nombre de comptes utilisateurs créés en 2020.\\
\textbf{Nombre de lignes attendues :} 1

\subsection*{Q14}
\textbf{Question :} Combien d’utilisateurs ont retweeté au moins une fois ?\\
\textbf{Nombre de lignes attendues :} 1 (avec la valeur 39)

\subsection*{Q15}
\textbf{Question :} Calculer combien d’utilisateurs n'ont pas fourni de biographie.\\
\textbf{Nombre de lignes attendues :} 1 (avec la valeur 8)

\subsection*{Q16}
\textbf{Question :} Lister les codes postaux et communes des utilisateurs dont le compte a été créé en 2020.\\
\textbf{Nombre de lignes attendues :} 5

\subsection*{Q17}
\textbf{Question :} Lister les tweets postés par les utilisatrices.\\
\textbf{Nombre de lignes attendues :} 110

\subsection*{Q18}
\textbf{Question :} Lister les codes postaux et noms de communes (sans doublon) des utilisateurs ayant retweeté au moins un tweet dont le texte commence par « Je ».\\
\textbf{Nombre de lignes attendues :} 12

\subsection*{Q19}
\textbf{Question :} Lister les pseudos des utilisateurs de Toulouse ayant posté un tweet en octobre 2021 qui a été retweeté.\\
\textbf{Nombre de lignes attendues :} 4

\subsection*{Q20}
\textbf{Question :} Lister les tweets qui ont été retweetés en 2021 par un utilisateur de Albi ou Castres.\\
\textbf{Nombre de lignes attendues :} 5

\subsection*{Q21}
\textbf{Question :} Lister les utilisateurs (pseudo+biographie) qui suivent \textit{AzurWorld}.\\
\textbf{Nombre de lignes attendues :} 4

\end{document}